
\chapter{Introduction}
\section{History}
Bandt and Pompe first paper about ordinal patterns is from 2002. In it they introduced the concept of turning time series data into patterns. They furthermore used Shannon entropy on the pattern distribution \cite{Bandt2002}. Shannon Entropy was developed in 1948 by C.E.Shannon \cite{Shannon1948}. 
\\\\
The 1995 paper "A statistical measure of complexity" by López-Ruiz, R. and Mancini, H. L. and Calbet, X introduces the concept of a statistical measure of complexity 
 \cite{LopezRuiz1995}. This idea was introduced into ordinal patterns in 2003 \cite{Martin2003} and in 2004 the version of it using Jensen-Shannon divergence were published \cite{Lamberti2004}, which is the version being used in this paper.

\section{Definition}
Given a time series $X=(X_1,X_2,...,X_{n+D-1})$. D is chosen, and recommend by Bandt and Pompe to be set between 3 and 7. D is the length of the subsequence each pattern represents. n is the number of patterns. A tuple of data points is transformed into a pattern by ranking them by numerical order. The lowest observation gets assigned the number 0 and the highest observation gets the number D-1. The pattern can then be written as a string of these numbers. A tuple $(1,3,2)$ will have pattern 021. 
\\\\
The exact order of each pattern becomes redundant, when calculating the entropy later, so it can be easier to think of the patterns simply as $\pi^1,\pi^2,..,\pi^{D!}$. Tuples that have the same ordering gets the same pattern, is the main point. There is D! different patterns. Equal values are ignored and can be offset by adding small random perturbations. The frequency of each pattern is defined as
$$p(\pi)=\frac{\#\{t|t\leq T-n,(x_{t+1},...,x_{t+n}) \mbox{ has type }\pi\}}{T-n+1}$$
\cite{Bandt2002}
\\\\
In the above section the delay time, $\tau$, between each observation in each subsequence, has been 1, however this can be set to any reasonable value that still produces enough patterns. In this paper only $\tau=1$ will be used.
\\\\\
From the pattern distribution entropies can be calculated. In this paper only Shannon entropy will be used. 
$$h(n)=-\sum p(\pi) log(p(\pi))$$
Normalized version
$$H(n)=-\frac{\sum p(\pi) ln(p(\pi))}{ln(D!)}$$
\\\\
There is several statistical complexity measures that can be used. They are all a product between the used entropy and a distance measure. In this paper Martin-Plastino-Rosso intesive Statistical Complexity Measure is used, where the distance measure is Jensen-Shannon divergence and it is measured between the pattern distribution and the uniform distribution. The $Q_0$ is normalizing term. It is defined as

$$C[\mathscr{P}]=H[\mathscr{P}]\cdot Q_J[\mathscr{P},\mathscr{P}_e]$$

$$Q_J[\mathscr{P},\mathscr{P}_e]=Q_0\cdot \mathscr{J}[\mathscr{P},\mathscr{P}_e]$$

$$\mathscr{J}[\mathscr{P},\mathscr{P}_e]=S[\frac{\mathscr{P}+\mathscr{P}_ee}{2}]-S[\frac{\mathscr{P}}{2}]-S[\frac{\mathscr{P}_e}{2}]$$

$$\mathscr{P}_e=\{p_j=\frac{1}{W};j=1...,W\}$$
$$Q_0 = -2(\frac{W+1}{W}ln(W+1)-2ln(2W)+ln(W))^{-1}$$
\cite{Amigo2023b}

\section{Applications}
Going to the original Bandt and Pompe paper \cite{Bandt2002} on Semantic Scholar and getting the ten most recent citations, gives a good picture of how broadly this methodology is being used.
\\\\
The topic varies from: "Walnut crack detection..." \cite{Zhang2024}, Schizophrenia \cite{Wang2024}, Analysis of Smart Drilling \cite{Szwajka2024}, Epileptic Seizure detection \cite{AbhishekParikh2024}, mind wandering during video-based learning \cite{Tang2024} and Random Numbers generated based on dual-channel chaotic light \cite{Liu2024}. The rest that was found \cite{Demirel2024, Du2024, Sun2024, Li2024}