\chapter{Conclusion}
Metadata about the health of reproducibility in the scientific field of Ordinal Patterns were gathered and analysed. Clear traces and indicators of a reproducibility crisis equal to that seen in many other scientific fields. The other main finding of the paper is the analysis of tie-breaking. To ensure stable results, across current implementations of the libraries in the field, it is necessary to both add noise and remove Na values. The new method proposed that breaks ties evenly among possible patterns were analysed. It has the benefit of not needing to add noise, making it a deterministic method, whereas adding noise is a stochastic method. It performed very well for the entropy and should be considered when dealing with a dataset, with a large degree of ties. For the p-values, it was slightly less impressive, since its p-values were not as close to the median of the 10 iterations. 