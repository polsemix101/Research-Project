\chapter{Comparison of time series}

\section{Examples of how comparisons are done in the literature}

Ordinal Patterns have been used in astrophysics to analyse geomagnetic auroras. 
Two articles were found to do this. 
They both compare the plotted points in the $H\times C$ plane to fractional Brownian motion time series. 
According to Weygand and Kivelson~\cite{Weygand2019}:
\begin{quote}
``Many of the points are on or very near the fractional Brownian motion curve, but a single point from the Helios data lies above the fractional Brownian motion curve.''
\end{quote}
A small amount of calculation is done in this paper to rate the statistical likelihood of a point being displaced from the fractional Brownian motion (fBm), but no method is presented to evaluate, if a point is displaced. 

The same issue appears in Osmane et al.~\cite{Osmane2019}:
\begin{quote}
``Figure 4 indicates that complexity-entropy values of AL overlap the fBm values for all subsampling parameters $\tau$.''
\end{quote}
No method is presented to systematically rate if a point overlaps or not with fBm values. 

Ordinal Patterns are often used in climate research to detect changes in the dynamics of a weather system over time. 
This is done, for instance, by Saco et al.~\cite{Saco2010}, by splitting the dataset into windows and calculating the entropy for each window. 
No method is presented to evaluate if a change in entropy is significant. 
It is assumed that a change in entropy means a change in the dynamics of the system, without considering the possibility that it might be caused by stochastic randomness of sampling. 

\section{Statistical Test and Confidence Intervals}

Confidence intervals describe the range within a sampling of a distribution has a certain percentage of being in. 
Statistical Test can be used to reject a wide range of hypotheses. 
The $p$-value is used as rejection criteria: the lower the $p$-values is, the lower is the chance of making a Type~I error, which is rejecting a true hypothesis~\cite{Smithson2003}.

\section{Confidence Interval of an Ordinal Pattern Distribution Entropy}

Rey et al.~\cite{Rey2023} obtained the asymptotic variance of the permutation entropy with the following rationale.
Assuming that $\textbf{X}_n = (X_{1,n},X_{2,n},...,X_{K,n})$ with $n \in \mathbb{N}$ is a sequence of independent and identically distributed $K$-variate vectors of random variables, and make $n$ tend to infinity.
Then, 
$$
\sqrt{n}(X_{1,n}-\theta_1,X_{2,n}-\theta_2,...,X_{K,n}-\theta_K)
$$ 
converges in distribution to the multivariate normal law $\mathscr{N}(\textbf{0},S_{\textbf{X}})$, where $S_{\textbf{X}}$ is the covariance matrix, $D$ is the embedding dimension, $\mathbf{q}=(q_1,q_2,...,q_{D!})$, where $q_i$ is the probability of observing the ordinal pattern $\pi_i$,
$\mathbf{D_q}=\text{Diag}(q_1,q_2,...,q_{D!})$ is a diagonal matrix, and
$\mathbf{Q}^{(\ell)}$ is the transition matrix with elements
$q_{ij}^{(\ell)}=\Pr(\psi =\pi_i \wedge \psi_{t+\ell}=\pi_j)$
for $\ell = 1,2,\dots,D-1$

From this starting point it is derived that 
\begin{gather*}
\sqrt{n}[S(\widehat{\textbf{q}})-S(\textbf{q})]  \overset{\mathscr{D}}{\underset{n\rightarrow\infty}{\longrightarrow}}\mathscr{N}(0,\sigma_{\textbf{q}}^2),
\intertext{where}
\sigma_{\textbf{q}}^2 = \sum_{i=1}^{D!}(\mathbf{\Sigma_q})_{ii}+2\sum_{i=1}^{D!-1}\sum_{j=i+1}^{D!}(\bm{\Sigma_q)}_{ij},
  (\mathbf{\Sigma_q})_{ij} = \left\{
\begin{array}{lr}
	(\ln q_i+1)^2\mathbf{\Sigma}_{ii} & \text{if $i=j$}\\
	(\ln q_i +1)(\ln q_j+1)\mathbf{\Sigma}_{ij} & \text{if $i \neq j$,}
\end{array}
\right.
\intertext{and}
\mathbf{\Sigma} = \mathbf{D_q}-(2m-1)\mathbf{qq}^T+\Sigma_{\ell=1}^{m-1}(\mathbf{Q}^{(\ell)}+{\mathbf{Q}^{(\ell)}}^T)
\end{gather*}
is the variance of interest, as it will be used for statistical tests. 

The test goes as follows. 
Let $x=(x_1,x_2,\dots,x_{n_x})$ and $y=(y_1,y_2,\dots,y_{n_y})$ be two independent time series of length $n_x=T_x+D_x-1$ and $n_y=T_y+D_y-1$. 
Denote $\bm p_x$ and $\bm p_y$ the ordinal distributions of the time series,
and $H(\widehat{\bm p}_x)$ and $H(\widehat{\bm p}_y)$ their entropies. 
A new distribution $W$ is constructed: 
$W \overset{D}{\rightarrow}\mathscr{N}(\mu_W,\sigma_W^2)$, with $\mu_W=\mu_{n_x,\bm p_x}-\mu_{n_y,\bm p_y}$ and $\sigma_W^2=\sigma_{n_x,\bm p_x}^2+\sigma_{n_y,\bm p_y}^2$. 
The $p$-value of the null hypothesis that the underlying entropies are equal ends up being $2(1-\Phi(\xi))$, where $\xi = \frac{H(\hat{p}_x)-H(\hat{p}_u)}{\sigma_W}$.

\section{Implementation}
The three main libraries that will be used specific to the field of ordinal patterns are: \texttt{statcomp}~\cite{statcomp}, \texttt{pdc}~\cite{pdc} and \texttt{StatOrdPattHxC}, which is a library developed by the supervisor of this project. 
Slight modifications are made to \texttt{StatOrdPattHxC}. 
As will be shown later, these three libraries perform quite differently in a lot of cases. Solutions for this are proposed. \texttt{StatOrdPattHxC} has the variance and statistical test implemented, which \texttt{statcomp} and \texttt{pdc} do not. 
A wide range of standard R libraries~\cite{RSoftware} is utilized as well.